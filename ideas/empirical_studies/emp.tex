%        File: summary.tex
%     Created: Thu Jul 12 07:00 PM 2012 H
% Last Change: Thu Jul 12 07:00 PM 2012 H
%
\documentclass[a4paper]{scrartcl}
\usepackage{hyperref}
\usepackage{bibentry}
\nobibliography*
\begin{document}
\title{Ideas on Empirical Studies in Opinion Formation}
\author{Mathis Antony}
\maketitle
\tableofcontents
\section{Problem}
Most models in Opinion Formation (OF) have individuals choose between two possible opinions.
The evolutionary aspect is introduced by randomly updating the opinions of individuals through some well defined model specific procedure.
In the spatial variant the individuals are placed on nodes of network and each individual interacts only with the ones situated on adjacent vertices.

The typical quantities of interest are the average opinion (often called ``magnetization'' by physicist) and the time it takes for the system to reach some form of equilibrium.
Thus in order to perform empirical studies we do need to gather data of the average opinion over time.
In addition to that, it would be interesting to also have information to have information about the underlying network structure or existing communities in the population.

Ideally, an opinion would not have as little effect on the persons behaviour as possible, like for example whether to vote for party A or party B in a political election.
The problem then is that such an opinion is difficult to capture without directly asking or observing the individual.
%
\section{Social Network}
With social network data it is rather straightforward to obtain information about the network structure but obtaining data about user's opinion may not be as easy.
We have thought of a few variables that may available and interesting for the study of OF:
\begin{itemize}
\item Choice of browser
\item Choice of OS
\item Choice of Architecture. Android or iOS? Does a person prefer the mobile app for a certain service or does she like to access it through the browser on her pc?
\end{itemize}
\section{In Class Experiments}
We have the okay from students in the class PHYS4058 Information Physics to ask them a few recurring questions during the semester.
We will also ask them to submit a list of who they interact with from time to time in order to get an idea on the network structure.
We are still looking for what we think are suitable questions.
As a first example, we will ask the students where on campus they ate their last meal.
This can then easily be mapped to a binary variable.
\end{document}

