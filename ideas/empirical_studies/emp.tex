%        File: summary.tex
%     Created: Thu Jul 12 07:00 PM 2012 H
% Last Change: Thu Jul 12 07:00 PM 2012 H
%
\documentclass[a4paper]{scrartcl}
\usepackage{hyperref}
\usepackage{bibentry}
\nobibliography*
\begin{document}
\title{Ideas on Empirical Studies in Opinion Formation}
\author{Mathis Antony}
\maketitle
\tableofcontents
\section{Problem}
Most models in Opinion Formation (OF) have individuals choose between two possible opinions.
The evolutionary aspect is introduced by randomly updating the opinions of individuals through some well defined model specific procedure.
In the spatial variant the individuals are placed on nodes of network and each individual interacts only with the ones situated on adjacent vertices.

The typical quantities of interest are the average opinion (often called ``magnetization'' by physicist) and the time it takes for the system to reach some form of equilibrium.
Thus in order to perform empirical studies we do need to gather data of the average opinion over time.
In addition to that, it would be interesting to also have information to have
information about the underlying network structure or existing communities in
the population. There are also many studies that obtained very interesting
results without introducing a network topology. The author of instance
---ref--- for instance explains via discussion group interaction how a
bias towards an opinion can lead to a population consenting on what was
initially the opinion of a minority.

To test the validity of some of the models developed by physicists an opinion
would not have as little effect on the person's experience as possible, like for
example whether to vote yes or no to a political issue. As opposed to the choice 
of a particular software, which may be entirely due to the user's experience
with the software, rather than the opinion of her peers.

Another problem lies within the difficulty to capture an opinion without directly 
asking or observing an individual. Because we have neither the funds nor the
manpower to conduct extensive studies over long periods of time. It may
therefore be more efficient to extract opinions social network data or other
sources.

%
\section{Social Networks}
With social network data it is rather straightforward to obtain information about the network structure but obtaining data about user's opinion may not be as easy.
We have thought of a few variables that may available and interesting for the study of OF:
\begin{itemize}
\item Choice of browser
\item Choice of OS
\item Choice of Architecture. Android or iOS? Does a person prefer the mobile app for a certain service or does she like to access it through the browser on her pc?
\end{itemize}
At this point I'm not really sure what sort of useful data from social networks
is available to us, therefore empirical studies with opinions on these subjects
may or may not be doable for us.

We could go a step further and extract hidden variables from content on social
networks. For instance posts on the microblogging website ``twitter''. This
adds additional complexity to our research as we'll have to learn how to use
tools created for knowledge extraction. As far as I know this is quite a hot
research topic itself and there should be plenty of available tools or at least
algorithms for knowledge extraction out there. On the upside this direction
would allow us to gather opinions on much more interesting topics such as
political issues instead of rather mundane topics such as choice of a
particular browser.

\section{Scanner Data}
The sociology professor we met suggest us to use scanner data to obtain
information about shopping behaviour. With this we could for example gather
opinions on what brand of a particular product people choose. Here the problem
is that we don't have any precise information about the network structure. It
may be possible to get a rough idea based on the geographical data of the
location of a particular shop or even the residential area where a shopper
lives. Here again I can imagine it to be rather tedious to obtain any data as I
see no reason for a supermarket chain to give out this data.

\section{In Class Experiments}
We have the okay from students in the class PHYS4058 Information Physics to ask them a few recurring questions during the semester.
We will also ask them to submit a list of who they interact with from time to time in order to get an idea on the network structure.
We are still looking for what we think are suitable questions.
As a first example, we will ask the students where on campus they ate their last meal.
This can then easily be mapped to a binary variable.
\end{document}

